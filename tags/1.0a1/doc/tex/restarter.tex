\section{\module{restarter} ---
         Automatically restart your program when module files change}

\declaremodule{lib537}{restarter}
\modulesynopsis{Automatically restart when module files change.}


This module solves the problem of refreshing Python modules in memory when the
source files change, without manually restarting the program. There are two
basic ways to solve this problem:

\begin{enumerate}

\item{\textbf{Reload modules within a single process.} The basic trick is to
delete items from \module{sys.modules}, forcing a refresh the next time they are
loaded. This gets tough because imported modules depend on each other. Look at
Zope and RollbackImporter for two implementations.}

\item{\textbf{Maintain two processes.} In this solution, a parent process
continuously restarts a child process. This gets around module import
dependencies, but the downside is that you loose any program state on restart,
and it magnifies the shutdown/start-up time of your program.}

\end{enumerate}

This module implements the second solution.

\subsection{Members}

\module{restarter} provides the following members:

\begin{datadesc}{CHILD, PARENT} These are booleans indicating whether the
current process is the parent or child. \end{datadesc}

\begin{funcdesc}{launch_child}{} Continuously relaunch the current program in a
sub-process until the exit code is something other than 75. \end{funcdesc}

\begin{funcdesc}{should_restart}{} Return a boolean indicating whether the child
process should be restarted. If called in the parent process, it always returns
\class{False}. \end{funcdesc}


Our implementation uses a thread in the child process (started when the module
is imported) to monitor all library source files. Your program is responsible
for periodically calling mods_changed, exiting with code 75 whenever it returns
\class{True} (presumably after cleanly shutting down). Exit code 75 seemed
appropriate to use because of its meaning on \UNIX{} systems (from
\file{/usr/include/sysexits.h}, FreeBSD 6.1-RELEASE):

\begin{verbatim}
EX_TEMPFAIL -- temporary failure, indicating something that
*              is not really an error.  In sendmail, this means
*              that a mailer (e.g.) could not create a connection,
*              and the request should be reattempted later.
[...]
#define EX_TEMPFAIL     75      /* temp failure; user is invited to retry */
\end{verbatim}

This module requires the \module{subprocess} module. Included in the standard
library since Python 2.4, it can also be found here:

\begin{verbatim}
http://www.lysator.liu.se/~astrand/popen5/
\end{verbatim}


\subsection{Example \label{example}}

Here is an example:

\begin{verbatim}
import restarter

def main():
    while 1:
        # program logic here
        if restarter.should_restart():
            # shutdown code here
            raise SystemExit(75)

if restarter.PARENT:
    restarter.launch_child()
else:
    main()
\end{verbatim}